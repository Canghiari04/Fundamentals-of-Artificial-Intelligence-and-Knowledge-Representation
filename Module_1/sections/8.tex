\textbf{Graph planning} is one of the most efficient type of complete and correct generative planning. It explicity introduces the concept of \textbf{time}, specifically \textbf{when} an
action is executed. This technique constructs a data structure called \textbf{planning graph}; the graph is extended at each step of the search process.
It has some crucial features:
\begin{itemize}[nosep]
    \renewcommand{\labelitemi}{-}
    \item \textbf{Shortest plan}. \\ Graphplan returns the shortest possible plan (the optimal one) from an initial state to a target state, or it indicates an inconsistency.
    \item \textbf{Closed world assumption}. \\ Graphplan uses the closed world assumption falling into the category of off-line planners.
    \item \textbf{Combining linear and non-linear planning.} \\ Graphplan inherits the \textbf{early commitment} feature from \textbf{linear planners} and the ability to create \textbf{partially ordered} sets of actions from \textbf{non-linear planners}.
\end{itemize}

As we saw in the STRIPS linear planning method, all the actions are defined by three different lists:
\begin{itemize}[nosep]
    \renewcommand{\labelitemi}{-}
    \item \textbf{Preconditions}: containing fluents that must be true for applying the action.
    \item \textbf{Delete}: fluents that become false after the execution phase.
    \item \textbf{Add}: fluents that become true after the execution. 
\end{itemize}

Sometimes the Add and Delete list are glued in the \textbf{Effect list}, which contains positive and negative literals. A particular characteristic of graph planning is that \textbf{objects}
have a \textbf{type}. For example, if we perform a general action $a$ to transition from state $X$ to state $Y$, $\langle X, a, Y \rangle$, the executed action 
will also describe the type of the object being manipulated in the new state. \vspace{3.5pt}

As we already know, the main issue in planning problems is the \textbf{frame problem}: \vspace{3.5pt}
\begin{center}
    \textit{Everything is untouched should hold in the next state}.
\end{center}

How does the graph planning strategy treat the frame problem? It uses a new type of action called the \textbf{no-op} (no-operation) action, that does not change the state. Let's consider an example
for better understanding.
\begin{example}
    i.e. Planning graph. \vspace{3.5pt}

    \begin{center}
        \includegraphics[width=0.4\textwidth]{img/img27.png}
    \end{center}

    The image above represents a \textbf{directed leveled graph}, divided into distinct levels. Adjacent levels are connected by \textbf{arcs}, and each of them contains a certain number
    of \textbf{nodes}. \vspace{3.5pt}

    Each level has its own meaning, and based on this example, we can observe the following structure:
    \begin{itemize}[nosep]
        \renewcommand{\labelitemi}{-}
        \item \textit{Level 0} corresponds to the initial state.
        \item \textit{Level 1} contains all actions whose preconditions are met by the propositions in \textit{level 0}.
        \item \textit{Level 2} contains all postconditions of the actions in \textit{level 1}.
        \item \dots
    \end{itemize} \vspace{3.5pt}

    The graph is further divided into two types of levels:
    \begin{itemize}[nosep]
        \renewcommand{\labelitemi}{-}
        \item \textbf{Propositional level}: nodes representing propositions, e.g. \textit{level 0}.
        \item \textbf{Action level}: nodes representing actions, e.g. \textit{level 1}.
    \end{itemize} \vspace{3.5pt}

    Arcs are also divided into categories:
    \begin{itemize}[nosep]
        \renewcommand{\labelitemi}{-}
        \item \textbf{Add arcs}: connect the action level to the propositional level.
        \item \textbf{Delete arcs}: as add arcs, connect the action level to the propositional level.
        \item \textbf{Preconditions arcs}: connect the propositional level to the action level.
    \end{itemize}
\end{example}

During the construction of the planning graph, various inconsistencies are idenfied, both for actions and propositions. Generally, actions and
propositions are \textbf{mutually exclusive}, therefore they cannot coexist together in a plan. However, a key feature of this method is that mutex actions or propositions
may initially appear in the same level of the planning graph. We must identify these inconsistencies and then apply a set of incompatibility rules. \vspace{3.5pt}

Two propositions are mutually exclusive if:
\begin{itemize}[nosep]
    \renewcommand{\labelitemi}{-}
    \item One proposition is the negation of the other.
    \item All the possible pairs of actions that could achieve them are mutually exclusive.
\end{itemize} \vspace{3.5pt}

For any two actions, there are three major types of inconsistencies:
\begin{itemize}[nosep]
    \renewcommand{\labelitemi}{-}
    \item \textbf{Interference}: one action deletes a precondition of the other.
    \item \textbf{Inconsistent effects}: one action negates the effect of another, e.g. the \textit{dash line} represented in the previous figure.
    \item \textbf{Competing needs}: two actions that have mutually exclusive preconditions.
\end{itemize} 

Once the principal incompatibilities have been identified, it's crucial to understand how the plannig graph is built. The process is composed by three main steps, which are:
\begin{itemize}[nosep]
    \renewcommand{\labelitemi}{}
    \item $1^{st}$. All the propositions cointained in the initial state are inserted in the \textbf{first propositional level}.
    \item $2^{nd}$. Create the \textbf{action level}: 
        \item \begin{itemize}[nosep]
            \renewcommand{\labelitemii}{}
            \item $2.1$ For every way to unify the previous propositions with the preconditions of the given operators, insert an \textbf{action node} in the subsequent \textbf{action level} if and only if two propositions are not labeled as \textbf{mutually exclusive}.
            \item $2.2$ For every proposition in the previous level add a \textbf{no-op} (no operation) node in the same \textbf{action level}.
            \item $2.3$ Check if the action nodes do not interfere each other, otherwise mark them as \textbf{mutually exclusive}.
        \end{itemize}
    \item $3^{th}$. Create the \textbf{propositional level}:
        \begin{itemize}[nosep]
            \renewcommand{\labelitemii}{}
            \item $3.1$ For each action node in the previous level, add a proposition node for every predicate described in the add and delete list of the operator. Propositions in the add list are linked by a \textbf{solid line}, instead the prepositions in the delete list are linked by a \textbf{dash line}.
            \item $3.2$ Do the same process for the no-op actions. Generally they are connected with a \textbf{solid line}.
            \item $3.3$ Mark as \textbf{mutually exclusive} two propositions such that all the ways to achieve the first are incompatible with all the ways to reach the second.
        \end{itemize}
\end{itemize}
\begin{example}
    i.e. Cart world. \vspace{7pt}

    We have a cart R and two loads A and B that are in the starting position L and must be moved to the target positions P\footnote{Here we can observe one of the peculiarities of the planning graph: all the objects taken into consideration are \textbf{typed}.}. \vspace{3.5pt}

    Given an initial state \vspace{7pt}
    \begin{center}
        \begin{tabular}{l}
            \textbf{Initial state} \\
            $at(A,L), at(L,B)$ \\
            $at(R,L), hasFuel(R)$
        \end{tabular}
    \end{center} \vspace{3.5pt}
    define a plan that reaches the goal state \vspace{7pt}
    \begin{center}
        \begin{tabular}{l}
            \textbf{Goal state} \\
            $at(A,P) \land at(B,P)$
        \end{tabular}
    \end{center} \vspace{7pt}

    The executable actions are:     
    \begin{itemize}[nosep]
        \renewcommand{\labelitemi}{}
        \item $move(R,Pos_X,Pos_Y)$
        \item Pre-conditions: 
            \begin{itemize}[nosep]
                \renewcommand{\labelitemii}{}
                \item $at(R,Pos_X), \textit{diff}(Pos_X,Pos_Y), hasFuel(R)$
            \end{itemize}
        \item Post-conditions: 
            \begin{itemize}[nosep]
                \renewcommand{\labelitemii}{}
                \item $at(R,Pos_Y), \neg at(R,Pos_X), \neg hasFuel(R)$
            \end{itemize} \vspace{7pt}

        \item $load(Object, Pos)$
        \item Pre-conditions: 
            \begin{itemize}[nosep]
                \renewcommand{\labelitemii}{}
                \item $at(R,Pos), at(Object,Pos)$
            \end{itemize}
        \item Post-conditions: 
            \begin{itemize}[nosep]
                \renewcommand{\labelitemii}{}
                \item $in(R,Object), \neg at(R,Pos), \neg at(Object,Pos)$
            \end{itemize} \vspace{7pt}

        \item $unload(Object, Pos)$
        \item Pre-conditions: 
            \begin{itemize}[nosep]
                \renewcommand{\labelitemii}{}
                \item $in(R,Object), at(R,Pos)$
            \end{itemize}
        \item Post-conditions: 
            \begin{itemize}[nosep]
                \renewcommand{\labelitemii}{}
                \item $at(Object,Pos), \neg in(R,Object)$
            \end{itemize} \vspace{7pt}
    \end{itemize} \vspace{7pt}

    $1^{st}$ step: define the first propositional level, including all the propositions in the given initial state. \vspace{3.5pt}
        \begin{center}
            \includegraphics[width=0.15\textwidth]{img/img28.png} 
        \end{center} \vspace{3.5pt}

    $2^{nd}$ step: create the action level, containing all the possible actions, according to the previous propositions, and the no-op nodes. \vspace{3.5pt}
        \begin{center}
            \includegraphics[width=0.25\textwidth]{img/img29.png} 
        \end{center} \vspace{3.5pt}

    $3^{th}$ step: describe in a new propositional level all the actions' preconditions and bring forward all the no-op nodes. \vspace{3.5pt}
        \begin{center}
            \includegraphics[width=0.5\textwidth]{img/img30.png} 
        \end{center}

    At the end, it's necessary to list all the incompatibilities dected by the planning graph, divided into: interference, inconsistent effects and competing needs. 

    \begin{center}
        \begin{tabular}{|l|}
            \hline
            \bf Interference \\ 
            \hline
            load(A, L) - move(R, L, P) \\
            load(B, L) - move(R, L, P) \\
            \hline
        \end{tabular} \vspace{3.5pt}
    \end{center} \vspace{7pt}

    \begin{center}
        \begin{tabular}{|l|}
            \hline
            \bf Inconsistent effects \\ 
            \hline
            load(A, L) - no-op at at(A, L) \\
            load(B, L) - no-op at at(B, L) \\
            move(R, L, P) - no-op at at(R, P) \\
            move(R, L, P) - no-op at hasFuel(R) \\
            \hline
        \end{tabular} \vspace{3.5pt}
    \end{center} \vspace{7pt}

    \begin{center}
        \begin{tabular}{|l|}
            \hline
            \bf Propositions incompatibilities \\ 
            \hline
            at(A, L) - in(R, A) \\ 
            at(B, L) - in(R, B) \\ 
            at(R, P) - at(R, L) \\ 
            at(R, P) - hasFuel(R) \\ 
            in(A, R) - at(R, P) \\ 
            in(B, R) - at(R, P) \\ 
            \hline
        \end{tabular} \vspace{3.5pt}
    \end{center} \vspace{7pt}

    Before moving on, it's crucial to focus on some steps done about the listed incompatibilities:
    \begin{itemize}[nosep]
        \renewcommand{\labelitemi}{-}
        \item \textbf{Interference}. \\ As we already know, interferences describe when an action deletes the precondition of an another action. Inside the planning graph this occurs between the predicates \textit{load(\dots)} and \textit{move(\dots)}.
        \item \textbf{Inconsistent Effects}. \\ Inconsistent effects define when an action negates the precondition of an another action. Generally, they happen within any possible actions executable in our domain and the \textit{no-op operators}.
        \item \textbf{Proprositions incompatibilities}. \\ Propositions incompatibilities represent a key concept in plannig graph, even though they have not been defined yet. We can see their as a summary of the all propositions incompatibilities already in the planning graph. Usually, the main incompatibilities are between: 
            \begin{itemize}[nosep]
                \item \textbf{No-op propositions / effects propositions}. 
                \item \textbf{New effects coming from the executed actions}. \\ For instance, we cannot have in the same propositional level the predicate $in(R, A)$ if the cart is already in the position $P$, $at(R, P)$.
            \end{itemize}
    \end{itemize}
\end{example}

Once the planning graph is built, we have to extract a \textbf{valid plan}, a connected and consistent subgraph of the planning graph\footnote{The \textbf{correctness} of
the valid plan chosen it's not guaranteed.}. The construction of the valid plan is done \textbf{backward}, in a opposite way according to the construction of 
the \textbf{planning graph}. \vspace{3.5pt}

Additionally, it's possible to take in account some useful theorems that allow us to judge if the valid plan retrieved by the planning graph it's at least partially correct.

\begin{definition}[title = $1^{st}$ Theorem]
    If there is a \textbf{valid plan} then it is a \textbf{subgraph} of the planning graph.
\end{definition}

\begin{definition}[title = $2^{nd}$ Theorem]
    In a planning graph two actions are \textbf{mutually exclusive} in a time step if a valid plan that contains both \textbf{does not exist}.
\end{definition}

\begin{definition}[title = $3^{rd}$ Theorem]
    In a planning graph two propositions are mutually exclusive in a time step if they are inconsistent, one of them \textbf{denies} the occurrence of the other\footnote{Given the block world, we cannot perform two \textbf{stack(\dots)} actions at the same time.}.
\end{definition}
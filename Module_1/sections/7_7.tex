Up to now we deal only with \textbf{generative planners}, also called offline planners: they assume a \textbf{closed world assumption}\footnote{Offline planners assume that nothing changes while they are planning.} to design a plan and finally 
execute the plan created. \vspace{3.5pt}

However, soo many problems should encountered during the execution of the actions:
\begin{itemize}[nosep]
    \renewcommand{\labelitemi}{-}
    \item Effects of an action are not the ones expected.
    \item An action should be executed but its preconditions are not satisfied.
\end{itemize} 

While planning or executing, the agent should perceive the changes in the world and act accordingly. \textbf{Reactive planners}, also called online planners,
execute real actions while they are planning; therefore they assume an \textbf{open world assumption}: they consider that the informations that are not 
explicitly stated are not false, but \textbf{unknown}. The unknown informations can be retrieved by \textbf{sensing actions}, modeled as any action seen so far.
One peculiarity of these new actions is that they do not change the current state, instead they are waiting to reach new informations by the world considered. \vspace{3.5pt}

There are different approaches, as \textbf{conditional planning} or \textbf{integration between planning and execution}; we explore the first type. Conditional planners 
are search algorithms that generate various alternative plans for each source of uncertainty of the plan. Basically, a conditional plan is composed by:
\begin{itemize}[nosep]
    \renewcommand{\labelitemi}{-}
    \item Casual actions, that change the current state.
    \item Sensing actions to retrieve unknown informations.
    \item Several alternative partial plans  of which only one will be executed depending on the results of the observations.
\end{itemize}
\begin{example}
    i.e. Tire world. \vspace{3.5pt}

    Given the initial state \vspace{7pt}
    \begin{center}
        \begin{tabular}{l}
            \textbf{Initial state} \\
            $on(tire1, hub1), flat(tire1),$ \\
            $\textit{inflated}(spare), \textit{off}(spare),$ \\
            $tire(tire1)$ 
        \end{tabular}
    \end{center} \vspace{3.5pt}

    define a plan that reaches the goal state \vspace{7pt}
    \begin{center}
        \begin{tabular}{l}
            \textbf{Goal state} \\
            $on(X,hub1) \land \textit{inflated}(X)$
        \end{tabular}
    \end{center}

    The executable actions are:  \vspace{7pt}
    \begin{center}
        \begin{tabular}{l}
            \textbf{putdown(X,Y)} \\
            Precond: $\textit{off}(X), clearHub(Y)$ \\
            Effect: $on(X,Y), \neg \textit{off}(X), \neg clearHub(Y)$
        \end{tabular} \vspace{7pt}

        \begin{tabular}{l}
            \textbf{remove(X,Y)} \\
            Precond: $on(X,Y), \neg intact(X)$ \\
            Effect: $\neg on(X,Y), \textit{off}(X), clearHub(Y)$
        \end{tabular} \vspace{7pt}
        
        \begin{tabular}{l}
            \textbf{check(X)}\footnote{\textbf{check(\dots)} is a sensing action. It does not change the current state but it retrives unknown informations from the world considered.} \\
            Precond: $tire(X)$ \\
            Effect: $knowsWhether(intact(X))$
        \end{tabular} \vspace{7pt}
        
        \begin{tabular}{l}
            \textbf{inflate(X)} \\
            Precond: $intact(X), flat(X)$ \\
            Effect: $\textit{inflated}(X), \neg flat(X)$
        \end{tabular} \vspace{7pt}
    \end{center} \vspace{3.5pt}

    \begin{center}
        \includegraphics[width=0.6\textwidth]{img/img31.png}
    \end{center} \vspace{3.5pt}

    The condition $intact(tire1)$ does not appear in the description of the initial state and cannot be obtained as effect of any action: the plan will certainly \textbf{fail}.
    Using the sensing action one we can build a \textbf{conditional plan}: \vspace{3.5pt}

    \begin{center}
        \includegraphics[width=0.7\textwidth]{img/img32.png}
    \end{center} \vspace{3.5pt}

    The initial partial plan $\textit{inflate}(tire1)$ is correct only if the condition $intact(tire1)$ is true in the initial state; if this is not verified, we should generate another partial plan
    that takes as precondition the opposite case, so the predicate $\neg intact(tire1)$. \vspace{3.5pt}

    \begin{center}
        \includegraphics[width=0.8\textwidth]{img/img33.png}
    \end{center} \vspace{3.5pt}

    We should generate a copy of the goal for every other executing scenario and create a corresponding plan for each of them. It's easy to observe that any conditional planner
    could generate an \textbf{exponential number} of plans.
\end{example}